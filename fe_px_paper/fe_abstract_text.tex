% An article usually includes an abstract, a concise summary of the work
% covered at length in the main body of the article. 
% % \begin{description}
% % \item[Usage]
% % Secondary publications and information retrieval purposes.
% 
% % Not using PACS anymore - deprecated in favor of PhySH now?
% % \item[PACS numbers]
% % May be entered using the \verb+\pacs{#1}+ command.
% % \item[Structure]
% % You may use the \texttt{description} environment to structure your abstract;
% use the optional argument of the \verb+\item+ command to give the category of each item. 
% % \end{description}


% % Intermediate-energy proton beams are used to produce a wide range of radionuclides for use in medical treatments and research.  
% % However, reaction modeling in this energy range remains largely untested, and there is a paucity of monitor reactions in this energy range needed to establish beam characteristics  for quantitative cross section measurements.  
% % To address this need, a
% A stack  of thin Fe 
% % , Cu, and Al monitor 
% foils was irradiated with  55\,MeV and 25\,MeV proton beams at Lawrence Berkeley National Laboratory's 88-Inch Cyclotron,  to investigate the \ce{^{nat}Fe}(p,x)  nuclear reaction channels as a  pathway for low-energy production of the emerging non-standard PET radionuclides \ce{^{51}Mn} and \ce{^{52m,g}Mn}.
% % monitor for intermediate energy proton experiments and to benchmark state-of-the-art reaction model codes.
% % A stacked target of natural Nb, Cu, and Al thin foils were irradiated by 100 MeV protons at the Los Alamos National Laboratory's Isotope Production Facility.
% % The irradiation was motivated by interest in the characterization of the \ce{^{93}Nb}(p,4n)\ce{^{90}Mo} reaction as a new monitor reaction standard for intermediate-energy protons.
% A set of 34 measured  cross sections for  \ce{^{nat}Fe}(p,x), \ce{^{nat}Cu}(p,x), and  \ce{^{nat}Ti}(p,x) reactions between threshold and 55\,MeV, as well as three independent measurements of isomer branching ratios, are reported. 
% % The measurements are 
% These are useful in medical and basic science radionuclide productions at low energies. 
% % Cross sections were measured using the well-characterized 
% The
% % \ce{^{nat}Al}(p,x)\ce{^{22}Na}, \ce{^{nat}Al}(p,x)\ce{^{24}Na}, 
% \ce{^{nat}Ti}(p,x)\ce{^{46}Sc}, \ce{^{nat}Ti}(p,x)\ce{^{48}V}, \ce{^{nat}Cu}(p,x)\ce{^{62}Zn}, and \ce{^{nat}Cu}(p,x)\ce{^{63}Zn} monitor  reactions were used to
% determine proton fluence, with all activities  quantified via HPGe spectrometry.
% Variance minimization techniques were employed to reduce systematic uncertainties in proton energy and fluence, improving the reliability of these measurements. 
% The measured cross sections are shown to be in excellent agreement with  literature values where they exist, and have been measured with improved precision compared with previous measurements.
% This work also reports the first measurement for $\leq$70\,MeV protons  of the  
% \ce{^{nat}Fe}(p,x)\ce{^{49}Cr},\ce{^{51,52m,56}Mn},\ce{^{58m}Co} reactions, as well as the first measurement of the independent cross sections for    \ce{^{nat}Fe}(p,x)\ce{^{51}Cr},\ce{^{52g}Mn},\ce{^{58g}Co}.
% % \textred{\ce{^{nat}Nb}(p,x)\ce{^{82m}Rb} reaction, and of the independent cross sections for    
% % \ce{^{nat}Cu}(p,x)\ce{^{52\text{m}}Mn}, 
% % \ce{^{nat}Cu}(p,x)\ce{^{52\text{g}}Mn} and \ce{^{nat}Nb}(p,x)\ce{^{85\text{g}}Y} in the 40--90\,MeV region.}
% % .
% % \textred{
% % Due to \ce{^{nat}Si}(p,x)\ce{^{22,24}Na} contamination  by  activation of the silicone adhesive used for sealing foil packets, the \ce{^{nat}Al}(p,x)\ce{^{22}Na} and \ce{^{nat}Al}(p,x)\ce{^{24}Na}  monitor reactions  were excluded, in the interest of surety, from proton fluence determination.
% % }
% % \textred{
% % To avoid inadvertently reporting enhanced proton fluence due to
% % The effects of  \ce{^{nat}Si}(p,x)\ce{^{22,24}Na} contamination, arising from  silicone adhesive in the Kapton tape used to encapsulate the aluminum monitor foils, is also discussed as a cautionary note to future stacked-target cross section measurements.
% % strongly discouraged.
% % }
% % Comparisons against the 
% \emph{A priori} predictions of the reaction modeling codes  CoH, EMPIRE, TALYS, and ALICE are compared with experimentally measured values and used to explore the differences between codes for the
% % impact of pre-equilibrium particle emission  for the 
% % 40--90 MeV 
% \ce{^{nat}Fe}(p,x), \ce{^{nat}Cu}(p,x), and  \ce{^{nat}Ti}(p,x) reactions, illustrating that these codes fail to reproduce the deposition of angular momentum at both compound and pre-compound energies.




Numerous reaction modeling packages can be used to calculate radionuclide production cross sections for medical, research, and national security applications. 
However, while all of these models invoke similar physical processes, their results often differ significantly.  
In this paper, we compare \emph{a priori} predictions from several state-of-the-art reaction models (CoH, EMPIRE, TALYS, and ALICE) to cross sections measured using the stacked-target activation method.  
The experiment was performed using the LBNL 88-Inch Cyclotron with beams of 25 and 55\,MeV protons on a stack of natural iron, copper, and titanium foils.  
A total of 34 excitation functions were measured for 4 \textless $E_p$ \textless 55\,MeV, including the first measurement of the independent cross sections for \ce{^{nat}Fe}(p,x)\ce{^{49,51}Cr}, \ce{^{51,52m,52g,56}Mn}, and \ce{^{58m,58g}Co}.  
All of the models failed to reproduce the isomer-to-ground state ratio for many  reaction channels at both compound and pre-compound energies, suggesting issues in modeling the deposition or distribution of angular momentum in these residual nuclei.  



% \begin{description}
% \item[Usage]
% Secondary publications and information retrieval purposes.
% \item[PACS numbers]
% May be entered using the \verb+\pacs{#1}+ command.
% \item[Structure]
% You may use the \texttt{description} environment to structure your abstract;
% use the optional argument of the \verb+\item+ command to give the category of each item. 
% \end{description}