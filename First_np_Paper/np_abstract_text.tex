
%\comment{Use DD or DD as convention?}



% \comment{Need Daniel's middle initial} - he doesn't have one



Cross sections for the \ce{^{47}Ti}(n,p)\ce{^{47}Sc} and \ce{^{64}Zn}(n,p)\ce{^{64}Cu} reactions have been measured for quasi-monoenergetic DD neutrons produced by the UC Berkeley High Flux Neutron Generator (HFNG).
The HFNG is a compact neutron generator designed as a \enquote{flux-trap} that maximizes the probability that a neutron will interact with a sample loaded into a specific, central location.  
The study was motivated by interest in the production of \ce{^{47}Sc} and \ce{^{64}Cu} as emerging medical isotopes.
The cross sections were measured in ratio to the \ce{^{113}In}(n,n')\ce{^{113m}In} and \ce{^{115}In}(n,n')\ce{^{115m}In} inelastic scattering reactions on co-irradiated indium samples.
Post-irradiation counting using an HPGe and LEPS detectors allowed for cross section determination to within 5\% uncertainty.
The \ce{^{64}Zn}(n,p)\ce{^{64}Cu} cross section for 2.76$^{+0.01}_{-0.02}$ MeV neutrons is reported as    49.3 $\pm$ 2.6 mb (relative to \ce{^{113}In}) or 46.4 $\pm$ 1.7 mb (relative to \ce{^{115}In}), and the \ce{^{47}Ti}(n,p)\ce{^{47}Sc} cross section is reported as 26.26 $\pm$  0.82 mb.
The measured cross sections  are found to be  in good agreement with existing measured values but with lower uncertainty (\textless 5\%), and also in agreement with  theoretical values.
This work highlights the utility of compact, flux-trap DD-based neutron sources for nuclear data measurements and potentially the production of radionuclides for medical applications.




% Cross sections for the \ce{^{47}Ti}(n,p)\ce{^{47}Sc} and \ce{^{64}Zn}(n,p)\ce{^{64}Cu} reactions have been measured for quasi-monoenergetic DD neutrons produced by the UC Berkeley High Flux Neutron Generator.
% The study was motivated by interest in the production of \ce{^{47}Sc} and \ce{^{64}Cu} as emerging medical isotopes.
% The cross sections were measured in ratio to the \ce{^{113}In}(n,n')\ce{^{113m}In} and \ce{^{115}In}(n,n')\ce{^{115m}In} inelastic scattering reactions on co-irradiated indium samples.
% Post-irradiation counting using an HPGe and LEPS detectors allowed for cross section determination to within 5\% uncertainty.
% The cross sections were determined with lower uncertainty than existing measurements and are found to be  in good agreement with both empirical and theoretical values.
% This work highlights the utility of using DD plasma-based neutron sources for a host of nuclear data measurements and potentially for the production of radionuclides for medical applications.

% \comment{Karl:  \enquote{Comment to engender some discussion.  I have a small concern here, reminiscent of what happened to our electron backstreaming paper in PRAB.  To be publishable, even in NIMB, there has to be a crisp research question resolved, or some innovation.  I think an angle that is missing here is that this is a new design of neutron generator, whose design maximizes the flux density (n/sec/cm2), although the total flux, while respectable, is not spectacular in itself.  This is Lee's recent mantra, and I now understand its significance.  Problematically, as Andrew has pointed out, we don't have the actual instrument paper published yet, so the thrust of the paper can't be too focused on the flux density issue, but a workable angle would be \enquote{given we have this new capability, this is an example of its power}.  Let me suggest Lee provide a sentence for the abstract, and a few sentences of text in the appropriate spot.}}

% \comment{The abstract is now nicely short and sweet, but should it be expanded at all?}
