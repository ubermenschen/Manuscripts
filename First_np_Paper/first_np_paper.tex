%%%%%%%%%%%%%%%%%%%%%%%%%%%%%%%%%%%%%%%%%%%%%%
%%%%%%%%%%%%%%%%%%%%%%%%%%%%%%%%%%%%%%%%%%%%%%
%%                                          %%
%% Important note on usage                  %%
%% -----------------------                  %%
%% This file must be compiled with PDFLaTeX %%
%% Using standard LaTeX will not work!      %%
%%                                          %%
%%%%%%%%%%%%%%%%%%%%%%%%%%%%%%%%%%%%%%%%%%%%%%
%%%%%%%%%%%%%%%%%%%%%%%%%%%%%%%%%%%%%%%%%%%%%%

%% The '3p' and 'times' class options of elsarticle are used for Elsevier CRC
\documentclass[5p]{elsarticle}
% \documentclass[3p,times]{elsarticle}
% \documentclass[3p]{elsarticle}

\usepackage[american]{babel}
\usepackage{amsmath}
\usepackage[version=3]{mhchem} 
% \usepackage{fixltx2e}
% \usepackage{refcount}
% \usepackage{siunitx}
% \usepackage{lastpage}
% \usepackage{textcomp}
\usepackage{mathtools}

\usepackage{xfrac}
\usepackage{lmodern}
\usepackage[hidelinks]{hyperref}
% \usepackage{cool}
% \usepackage{cancel}
\usepackage{microtype}
\usepackage{listings}
\usepackage{mcode}
\usepackage [autostyle, english = american]{csquotes}
\usepackage{longtable}
% \usepackage{subcaption}
\usepackage{booktabs,siunitx}
\usepackage{gensymb}
\usepackage[normalem]{ulem}

% \usepackage{mathtools, cuted}


% \usepackage[usenames,dvipsnames,svgnames,table]{xcolor}
\usepackage{color}

\usepackage[colorinlistoftodos]{todonotes}

\usepackage[section]{placeins}
\usepackage{multirows}


\lstset{basicstyle=\small\ttfamily,columns=fullflexible}

% \usepackage{verbatim}



% \usepackage{gensymb}
% \usepackage{enumerate}
% \usepackage{float}
% \usepackage{bm}
% \usepackage{csquotes}
% \usepackage{mathtools}
% \usepackage{natbib}
% \usepackage{biblatex}

%% The `ecrc' package must be called to make the CRC functionality available
\usepackage{ecrc}

%% The ecrc package defines commands needed for running heads and logos.
%% For running heads, you can set the journal name, the volume, the starting page and the authors

%% set the volume if you know. Otherwise `00'
\volume{00}

%% set the starting page if not 1
\firstpage{1}

%% Give the name of the journal
\journalname{Nuclear Instruments and Methods in Physics Research B}

%% Give the author list to appear in the running head
%% Example \runauth{C.V. Radhakrishnan et al.}
\runauth{A.S. Voyles et al.}

%% The choice of journal logo is determined by the \jid and \jnltitlelogo commands.
%% A user-supplied logo with the name <\jid>logo.pdf will be inserted if present.
%% e.g. if \jid{yspmi} the system will look for a file yspmilogo.pdf
%% Otherwise the content of \jnltitlelogo will be set between horizontal lines as a default logo

%% Give the abbreviation of the Journal.
\jid{nimb}
% \jid{yspmi}

%% Give a short journal name for the dummy logo (if needed)
\jnltitlelogo{Nucl Instrum Meth B}

%% Hereafter the template follows `elsarticle'.
%% For more details see the existing template files elsarticle-template-harv.tex and elsarticle-template-num.tex.

%% Elsevier CRC generally uses a numbered reference style
%% For this, the conventions of elsarticle-template-num.tex should be followed (included below)
%% If using BibTeX, use the style file elsarticle-num.bst

%% End of ecrc-specific commands
%%%%%%%%%%%%%%%%%%%%%%%%%%%%%%%%%%%%%%%%%%%%%%%%%%%%%%%%%%%%%%%%%%%%%%%%%%

%% The amssymb package provides various useful mathematical symbols
\usepackage{amssymb}
%% The amsthm package provides extended theorem environments
\usepackage{amsthm}

%% The lineno packages adds line numbers. Start line numbering with
%% \begin{linenumbers}, end it with \end{linenumbers}. Or switch it on
%% for the whole article with \linenumbers after \end{frontmatter}.
%% \usepackage{lineno}

%% natbib.sty is loaded by default. However, natbib options can be
%% provided with \biboptions{...} command. Following options are
%% valid:

%%   round  -  round parentheses are used (default)
%%   square -  square brackets are used   [option]
%%   curly  -  curly braces are used      {option}
%%   angle  -  angle brackets are used    <option>
%%   semicolon  -  multiple citations separated by semi-colon
%%   colon  - same as semicolon, an earlier confusion
%%   comma  -  separated by comma
%%   numbers-  selects numerical citations
%%   super  -  numerical citations as superscripts
%%   sort   -  sorts multiple citations according to order in ref. list
%%   sort&compress   -  like sort, but also compresses numerical citations
%%   compress - compresses without sorting
%%
%% \biboptions{comma,round}

% \biboptions{}

% if you have landscape tables
\usepackage[figuresright]{rotating}

% put your own definitions here:
%   \newcommand{\cZ}{\cal{Z}}
%   \newtheorem{def}{Definition}[section]
%   ...

% add words to TeX's hyphenation exception list
%\hyphenation{author another created financial paper re-commend-ed Post-Script}

% declarations for front matter

\usepackage{fancyvrb}
\usepackage{color}
 
\definecolor{mygreen}{rgb}{0,0.6,0}
\definecolor{mygray}{rgb}{0.5,0.5,0.5}
\definecolor{mymauve}{rgb}{0.58,0,0.82}

\lstset{ %
  backgroundcolor=\color{white},   % choose the background color
  basicstyle=\footnotesize,        % size of fonts used for the code
  breaklines=true,                 % automatic line breaking only at whitespace
  captionpos=b,                    % sets the caption-position to bottom
  commentstyle=\color{mygreen},    % comment style
  escapeinside={\%*}{*)},          % if you want to add LaTeX within your code
  keywordstyle=\color{blue},       % keyword style
  stringstyle=\color{mymauve},     % string literal style
}

% Sin and Cos with auto-parentheses 
\newcommand{\sinp}[1]{\sin{\left( #1\right)}}
\newcommand{\cosp}[1]{\cos{\left( #1\right)}}
\newcommand{\expp}[1]{\exp{\left( #1\right)}}
\newcommand{\sinhp}[1]{\sinh{\left( #1\right)}}
\newcommand{\lnp}[1]{\ln{\left( #1\right)}}
\newcommand{\pp}[1]{\left( #1\right)}
\newcommand{\sci}[2]{ #1 \cdot 10^{#2}\ }
\newcommand{\angstrom}{\mbox{\normalfont\AA}}
\newcommand{\norm}[1]{\lVert #1 \rVert}

\newcommand{\textred}[1]{\textcolor{red}{ #1}}
\newcommand{\redactedit}[1]{\textcolor{blue}{ \sout{#1}}}


\newcommand{\colornote}[1]{\textcolor{red}{ COMMENT\large\footnote{\textcolor{red}{#1}}}}

\newcommand{\comment}[1]{\todo[color=blue!20!white,inline]{ASV: #1}} 

% Tweak sim for better inline text tilde
\newcommand{\mytilde}{\raisebox{0.5ex}{\texttildelow}}
% \newcommand{\mytilde}{\raise.17ex\hbox{$\scriptstyle‌​\sim$}}

\sisetup{separate-uncertainty=true,table-space-text-post = *}

\newcommand{\minitab}[2][l]{\begin{tabular}{#1}#2\end{tabular}}


\newcommand{\subfigimg}[4][,]{%
  \setbox1=\hbox{\includegraphics[#1]{#3}}% Store image in box
  \leavevmode\rlap{\usebox1}% Print image
  \rlap{\hspace*{#4pt}\raisebox{\dimexpr\ht1-2\baselineskip}{#2}}% Print label
  \phantom{\usebox1}% Insert appropriate spcing
}
% Old version of macro
% \newcommand{\subfigimg}[3][,]{%
%   \setbox1=\hbox{\includegraphics[#1]{#3}}% Store image in box
%   \leavevmode\rlap{\usebox1}% Print image
%   \rlap{\hspace*{50pt}\raisebox{\dimexpr\ht1-2\baselineskip}{#2}}% Print label
%   \phantom{\usebox1}% Insert appropriate spcing
% }
\usepackage{subfig}
% Remove a), b), etc labels from subfigs
\captionsetup[subfigure]{labelformat=empty}



\makeatletter
% Make common definition of mean
\newcommand*\mean[1]{\overline{#1\raisebox{3mm}{}}}

\makeatother


\begin{document}

\begin{frontmatter}

%% Title, authors and addresses

%% use the tnoteref command within \title for footnotes;
%% use the tnotetext command for the associated footnote;
%% use the fnref command within \author or \address for footnotes;
%% use the fntext command for the associated footnote;
%% use the corref command within \author for corresponding author footnotes;
%% use the cortext command for the associated footnote;
%% use the ead command for the email address,
%% and the form \ead[url] for the home page:
%%
\title{Measurement of the \ce{^{64}Zn},\ce{^{47}Ti}(n,p) Cross Sections using a DD Neutron Generator for Medical Isotope Studies}

% \title{(n,p) Cross Section Measurements for \ce{^{64}Cu} and \ce{^{47}Sc} via DD Neutron Generator\tnoteref{label1}}
%% \tnotetext[label1]{}
%% \author{Name\corref{cor1}\fnref{label2}}
%% \ead{email address}
%% \ead[url]{home page}
%% \fntext[label2]{}
%% \cortext[cor1]{}
%% \address{Address\fnref{label3}}
%% \fntext[label3]{}

% \dochead{Short}
%% Use \dochead if there is an article header, e.g. \dochead{Short communication}

% \title{NE290A - Homework \#04}
% \date{29 April, 2016}

% \author[rvt]{C.V. ̃Radhakrishnan\corref{cor1}\fnref{fn1}}
\author[ucb]{A.S. Voyles \corref{cor1}}
\ead{andrew.voyles@berkeley.edu}

\author[lbl]{M.S. Basunia}

\author[ucb]{J.C. Batchelder}

\author[llnl]{J.D. Bauer}

\author[geo]{T.A. Becker}


\author[ucb,lbl]{L.A. Bernstein}


\author[ucb]{E.F. Matthews}

\author[geo,eps]{P.R. Renne}

\author[geo,eps]{D. Rutte}

\author[ucb]{M.A. Unzueta}

\author[ucb]{K.A. van Bibber}



% \author[lbl]{K. ̃Bazargan\fnref{fn2}}
% \ead{kaveh@river-valley.com}



%% use optional labels to link authors explicitly to addresses:
%% \author[label1,label2]{<author name>}
%% \address[label1]{<address>}
%% \address[label2]{<address>}

\cortext[cor1]{Corresponding author}
% \cortext[cor2]{Principal corresponding author}
% \fntext[fn1]{This is the specimen author footnote.}
% \fntext[fn2]{Another author footnote, but a little more longer.}

% \address[ucb]{Department of Nuclear Engineering, University of California, Berkeley, Etcheverry Hall, 2521 Hearst Ave, Berkeley, CA 94709}
% \address[lbl]{Lawrence Berkeley National Laboratory,  1 Cyclotron Rd, Berkeley, CA 94720}
% \address[llnl]{Lawrence Livermore National Laboratory, 7000 East Ave, Livermore, CA 94550}

\address[ucb]{Department of Nuclear Engineering, University of California, Berkeley, Berkeley CA, 94720 USA}
\address[lbl]{Lawrence Berkeley National Laboratory,  Berkeley CA, 94720 USA}
\address[llnl]{Lawrence Livermore National Laboratory, Livermore CA, 94551 USA}
\address[geo]{Berkeley Geochronology Center, Berkeley CA,  94709  USA}
\address[eps]{Department of Earth and Planetary Sciences, University of California, Berkeley, Berkeley CA,  94720  USA}



% \author{Andrew Voyles}

% \address{Nuclear Engineering, University of California Berkeley, Berkeley, CA 94709, USA.}



\begin{abstract}


%\comment{Use DD or DD as convention?}



% \comment{Need Daniel's middle initial} - he doesn't have one



Cross sections for the \ce{^{47}Ti}(n,p)\ce{^{47}Sc} and \ce{^{64}Zn}(n,p)\ce{^{64}Cu} reactions have been measured for quasi-monoenergetic DD neutrons produced by the UC Berkeley High Flux Neutron Generator (HFNG).
The HFNG is a compact neutron generator designed as a \enquote{flux-trap} that maximizes the probability that a neutron will interact with a sample loaded into a specific, central location.  
The study was motivated by interest in the production of \ce{^{47}Sc} and \ce{^{64}Cu} as emerging medical isotopes.
The cross sections were measured in ratio to the \ce{^{113}In}(n,n')\ce{^{113m}In} and \ce{^{115}In}(n,n')\ce{^{115m}In} inelastic scattering reactions on co-irradiated indium samples.
Post-irradiation counting using an HPGe and LEPS detectors allowed for cross section determination to within 5\% uncertainty.
The \ce{^{64}Zn}(n,p)\ce{^{64}Cu} cross section for 2.76$^{+0.01}_{-0.02}$ MeV neutrons is reported as    49.3 $\pm$ 2.6 mb (relative to \ce{^{113}In}) or 46.4 $\pm$ 1.7 mb (relative to \ce{^{115}In}), and the \ce{^{47}Ti}(n,p)\ce{^{47}Sc} cross section is reported as 26.26 $\pm$  0.82 mb.
The measured cross sections  are found to be  in good agreement with existing measured values but with lower uncertainty (\textless 5\%), and also in agreement with  theoretical values.
This work highlights the utility of compact, flux-trap DD-based neutron sources for nuclear data measurements and potentially the production of radionuclides for medical applications.




% Cross sections for the \ce{^{47}Ti}(n,p)\ce{^{47}Sc} and \ce{^{64}Zn}(n,p)\ce{^{64}Cu} reactions have been measured for quasi-monoenergetic DD neutrons produced by the UC Berkeley High Flux Neutron Generator.
% The study was motivated by interest in the production of \ce{^{47}Sc} and \ce{^{64}Cu} as emerging medical isotopes.
% The cross sections were measured in ratio to the \ce{^{113}In}(n,n')\ce{^{113m}In} and \ce{^{115}In}(n,n')\ce{^{115m}In} inelastic scattering reactions on co-irradiated indium samples.
% Post-irradiation counting using an HPGe and LEPS detectors allowed for cross section determination to within 5\% uncertainty.
% The cross sections were determined with lower uncertainty than existing measurements and are found to be  in good agreement with both empirical and theoretical values.
% This work highlights the utility of using DD plasma-based neutron sources for a host of nuclear data measurements and potentially for the production of radionuclides for medical applications.

% \comment{Karl:  \enquote{Comment to engender some discussion.  I have a small concern here, reminiscent of what happened to our electron backstreaming paper in PRAB.  To be publishable, even in NIMB, there has to be a crisp research question resolved, or some innovation.  I think an angle that is missing here is that this is a new design of neutron generator, whose design maximizes the flux density (n/sec/cm2), although the total flux, while respectable, is not spectacular in itself.  This is Lee's recent mantra, and I now understand its significance.  Problematically, as Andrew has pointed out, we don't have the actual instrument paper published yet, so the thrust of the paper can't be too focused on the flux density issue, but a workable angle would be \enquote{given we have this new capability, this is an example of its power}.  Let me suggest Lee provide a sentence for the abstract, and a few sentences of text in the appropriate spot.}}

% \comment{The abstract is now nicely short and sweet, but should it be expanded at all?}



\end{abstract}

\begin{keyword}
%% keywords here, in the form: keyword \sep keyword
DD neutron generator \sep Medical Isotope Production \sep Scandium (Sc) and Copper (Cu) radioisotopes \sep Indium \sep Ratio activation \sep Theranostics

%% MSC codes here, in the form: \MSC code \sep code
%% or \MSC[2008] code \sep code (2000 is the default)

\end{keyword}

\end{frontmatter}

%%
%% Start line numbering here if you want
%%
% \linenumbers

% \listoftodos


%% main text 
% \include{./np_body_text}
\input{./np_body_text}
 
%  \comment{Should we specifically mention other isotopes we plan to measure?}


 
 \section{Acknowledgements}
 
 We would like to particularly point out the crucial role played by Cory Waltz in the design and commissioning of the HFNG.
 We wish to thank Marc Garland and Saed Mirzadeh for discussions regarding the use of neutron generators for isotope production.
 We acknowledge Glenn Jones of G\&J Jones Enterprises of Dublin, CA for the construction  of the High Flux Neutron Generator. 
 Lastly, we would like to acknowledge the students in the Nuclear Reactions and Radiation (NE102) laboratory course at UC Berkeley who participated in these experiments, including Joe Corvino, Nizelle Fajardo, Scott Parker and Evan Still.  
 
 This work has been carried out at the University of California, Berkeley, and performed under the auspices of the U.S. Department of Energy by Lawrence Livermore National Laboratory under contract \# DE-AC52-07NA27344 and Lawrence Berkeley National Laboratory under contract \# DE-AC02-05CH11231.
Funding has been provided from the US Nuclear Regulatory Commission, the US Nuclear Data Program, the Berkeley Geochronology Center, NSF ARRA Grant \# EAR-0960138, the University of California Laboratory Fees Research Grant \# 12-LR-238745, and  DFG Research Fellowship \# RU 2065/1-1.

% \comment{Add NSF Grant number from P. Renne.}
 



% \pagebreak
% 
% \onecolumn
% 
% \appendix


% \section{Data} \label{data}
% 
% % \begin{longtable}{|c|c|c|c|c|} 
% % \caption{My caption}
% % \label{tab:dummy}
% % % \begin{tabular}{|c|c|c|c|c|}
% %    \hline
% %  
% % \end{longtable}
% hello world 

% 
% \section{Results} \label{equations}
% 
% Tabulated results are listed here, to avoid breaking text flow in the main text.
% 
% \begin{table}[h]
% \centering
% \caption{Transition conversion coefficients and efficiencies}
% \label{my-label2}
% \begin{tabular}{c|c|c|c|c}
% Nucleus & Transition       & E\(_\gamma\) (keV) & \(\alpha\), Conversion Coefficient  & \(\epsilon\), Intrinsic Efficiency \\ \hline
% \ce{^{110}Ru}   & 2\(\rightarrow\)0 & 240.7          & 0.0569           \(\pm\) 0.0008              & 1863.906   \\
%         & 4\(\rightarrow\)2 & 422.6          & 0.00887          \(\pm\) 0.00013             & 1219.507   \\
%         & 6\(\rightarrow\)4 & 575.7          & 0.00356          \(\pm\) 0.00005             & 956.569    \\ \hline
% \ce{^{140}Xe}   & 2\(\rightarrow\)0 & 376.657        & 0.0205          \(\pm\) 0.0003              & 1340.054   \\
%         & 4\(\rightarrow\)2 & 457.63         & 0.01154          \(\pm\) 0.00017             & 1142.505   \\
%         & 6\(\rightarrow\)4 & 582.4          & 0.00593          \(\pm\) 0.00009             & 948.705    \\ \hline
% \ce{^{140}Ba}   & 2\(\rightarrow\)0 & 602.35         & 0.00599         \(\pm\) 0.00009             & 926.624    \\
%         & 4\(\rightarrow\)2 & 528.25         & 0.00848         \(\pm\) 0.00012             & 1019.604   \\
%         & 6\(\rightarrow\)4 & 529.7          & 0.00842         \(\pm\) 0.00012             & 1017.470   \\ \hline
% \ce{^{144}Ba}   & 2\(\rightarrow\)0 & 199.326        & 0.175           \(\pm\) 0.0025              & 2085.452   \\
%         & 4\(\rightarrow\)2 & 330.88         & 0.0333         \(\pm\) 0.0005              & 1485.782   \\
%         & 6\(\rightarrow\)4 & 431.3          & 0.015          \(\pm\) 0.00021             & 1199.258  
% \end{tabular}
% \end{table}
% 
% \begin{table}[h]
% \centering
% \caption{Extracted transition intensities}
% \label{my-label}
% \begin{tabular}{c|c|c|c|c|c}
% Nucleus & Transition       & E\(_\gamma\) (keV) & \(A_{obs}\), Peak Area (a.u.)   & \(A_{source}\),  Peak Intensity (a.u.)  & \(R_i\), Relative Intensity  \\ \hline
% \ce{^{110}Ru}   & 2\(\rightarrow\)0 & 240.7          & 1097    \(\pm\)   136        & 0.622          \(\pm\) 0.077             & 1.000              \(\pm\) 0.175                 \\
%         & 4\(\rightarrow\)2 & 422.6          & 515     \(\pm\)   107        & 0.426          \(\pm\) 0.089             & 0.685              \(\pm\) 0.200                 \\
%         & 6\(\rightarrow\)4 & 575.7          & 144      \(\pm\)  55         & 0.151          \(\pm\) 0.058             & 0.243              \(\pm\) 0.198                 \\ \hline
% \ce{^{140}Xe}   & 2\(\rightarrow\)0 & 376.657        & 542     \(\pm\)   129        & 0.413          \(\pm\) 0.098             & 1.000              \(\pm\) 0.337                 \\
%         & 4\(\rightarrow\)2 & 457.63         & 357       \(\pm\) 59         & 0.316          \(\pm\) 0.052             & 0.766              \(\pm\) 0.254                 \\
%         & 6\(\rightarrow\)4 & 582.4          & 161       \(\pm\) 61         & 0.171          \(\pm\) 0.065             & 0.414              \(\pm\) 0.288                 \\ \hline
% \ce{^{140}Ba}   & 2\(\rightarrow\)0 & 602.35         & 822       \(\pm\) 164        & 0.892          \(\pm\) 0.178             & 1.000              \(\pm\) 0.282                 \\
%         & 4\(\rightarrow\)2 & 528.25         & 384       \(\pm\) 78         & 0.380          \(\pm\) 0.077             & 0.426              \(\pm\) 0.186                 \\
%         & 6\(\rightarrow\)4 & 529.7          & 310       \(\pm\) 104        & 0.307          \(\pm\) 0.103             & 0.344              \(\pm\) 0.229                 \\ \hline
% \ce{^{144}Ba}   & 2\(\rightarrow\)0 & 199.326        & 900       \(\pm\) 242        & 0.507          \(\pm\) 0.136             & 1.000              \(\pm\) 0.380                 \\
%         & 4\(\rightarrow\)2 & 330.88         & 535       \(\pm\) 127        & 0.372          \(\pm\) 0.088             & 0.734              \(\pm\) 0.307                 \\
%         & 6\(\rightarrow\)4 & 431.3          & 363       \(\pm\) 64         & 0.307          \(\pm\) 0.054             & 0.606              \(\pm\) 0.250                
% \end{tabular}
% \end{table}
%     
% \pagebreak
% 
% 
% 
% \section{Supplemental Figures} \label{fit_figures}
% 
% The remaining figures of the fitted \(\gamma\)-ray transition photopeaks are included here for reference, to avoid over-cluttering the main text.
% 
% 
% % \begin{figure}[h]
% %  \centering
% %  \includegraphics[scale=0.7]{./hw04/fit110Ru240-cropped.pdf}
% %  % fit110Ru240.ps: 570x750 pixel, 72dpi, 20.11x26.46 cm, bb=0 0 570 750
% %  \caption{Fit to the \ce{^{110}Ru} 240.7 keV peak and its surroundings.}
% %  \label{fig:110Ru240}
% % \end{figure}
% % 
% % \begin{figure}[h]
% %  \centering
% %  \includegraphics[scale=0.7]{./hw04/fit110Ru422-cropped.pdf}
% %  % fit110Ru240.ps: 570x750 pixel, 72dpi, 20.11x26.46 cm, bb=0 0 570 750
% %  \caption{Fit to the \ce{^{110}Ru} 422.6 keV peak and its surroundings.}
% %  \label{fig:110Ru422}
% % \end{figure}
% % 
% % \begin{figure}[h]
% %  \centering
% %  \includegraphics[scale=0.7]{./hw04/fit110Ru575-cropped.pdf}
% %  % fit110Ru240.ps: 570x750 pixel, 72dpi, 20.11x26.46 cm, bb=0 0 570 750
% %  \caption{Fit to the \ce{^{110}Ru} 575.7 keV peak and its surroundings.}
% %  \label{fig:110Ru575}
% % \end{figure}
% 
% 
% 
% 
% % \begin{figure}[h]
% %  \centering
% %  \includegraphics[scale=0.7]{./hw04/fit140Xe376-cropped.pdf}
% %  % fit110Ru240.ps: 570x750 pixel, 72dpi, 20.11x26.46 cm, bb=0 0 570 750
% %  \caption{Fit to the \ce{^{140}Xe} 376.7 keV peak and its surroundings.}
% %  \label{fig:140Xe376}
% % \end{figure}
% % 
% % \begin{figure}[h]
% %  \centering
% %  \includegraphics[scale=0.7]{./hw04/fit140Xe457-cropped.pdf}
% %  % fit110Ru240.ps: 570x750 pixel, 72dpi, 20.11x26.46 cm, bb=0 0 570 750
% %  \caption{Fit to the \ce{^{140}Xe} 457.6 keV peak and its surroundings.}
% %  \label{fig:140Xe457}
% % \end{figure}
% % 
% % \begin{figure}[h]
% %  \centering
% %  \includegraphics[scale=0.7]{./hw04/fit140Xe582-cropped.pdf}
% %  % fit110Ru240.ps: 570x750 pixel, 72dpi, 20.11x26.46 cm, bb=0 0 570 750
% %  \caption{Fit to the \ce{^{140}Xe} 582.3 keV peak and its surroundings.}
% %  \label{fig:140Xe582}
% % \end{figure}
% % 
% % 
% % 
% % \begin{figure}[h]
% %  \centering
% %  \includegraphics[scale=0.7]{./hw04/fit140Ba602-cropped.pdf}
% %  % fit110Ru240.ps: 570x750 pixel, 72dpi, 20.11x26.46 cm, bb=0 0 570 750
% %  \caption{Fit to the \ce{^{140}Ba} 602.4 keV peak and its surroundings.}
% %  \label{fig:140Ba602}
% % \end{figure}
% % 
% % 
% % 
% % 
% % 
% % \begin{figure}[h]
% %  \centering
% %  \includegraphics[scale=0.7]{./hw04/fit144Ba199-cropped.pdf}
% %  % fit110Ru240.ps: 570x750 pixel, 72dpi, 20.11x26.46 cm, bb=0 0 570 750
% %  \caption{Fit to the \ce{^{144}Ba} 199.3 keV peak and its surroundings.}
% %  \label{fig:144Ba199}
% % \end{figure}
% % 
% % 
% % 
% % \begin{figure}[h]
% %  \centering
% %  \includegraphics[scale=0.7]{./hw04/fit144Ba330-cropped.pdf}
% %  % fit110Ru240.ps: 570x750 pixel, 72dpi, 20.11x26.46 cm, bb=0 0 570 750
% %  \caption{Fit to the \ce{^{144}Ba} 330.9 keV peak and its surroundings.}
% %  \label{fig:144Ba330}
% % \end{figure}
% % 
% % \begin{figure}[h]
% %  \centering
% %  \includegraphics[scale=0.7]{./hw04/fit144Ba431-cropped.pdf}
% %  % fit110Ru240.ps: 570x750 pixel, 72dpi, 20.11x26.46 cm, bb=0 0 570 750
% %  \caption{Fit to the \ce{^{144}Ba} 431.3 keV peak and its surroundings.}
% %  \label{fig:144Ba431}
% % \end{figure}
% 
% 
% 
%     
% \pagebreak
% 
% 
% % \appendix
% \section{Source Code} \label{source}
% 
% \begin{center}
%  \texttt{gint.py} - Implementation of the Wilhelmy statistical model, used to estimate total transition intensities in the ground state band  of primary fission fragments
% \end{center}
% 
% % \begin{lstlisting}
% % \lstloadlanguages{Python}
% 
% % \lstinputlisting[linewidth=\columnwidth,language=Python,breaklines=true,frame=single]{./hw04/gint.py}
% % % \end{lstlisting}
% 
% 
% 
% \begin{center}
%  \texttt{hw04\_04.m} - used to process and plot statistical transition intensity curves
% \end{center}
% 
%  
% %   \lstinputlisting[frame=single]{./hw04/hw04_04.m}
% 
% 
% % \begin{verbatim}[commandchars=\\\{\}]
% % \verbatiminput{./HW01/test.txt}
% % \end{verbatim}
% 
% %% The Appendices part is started with the command \appendix;
% %% appendix sections are then done as normal sections
% %% \appendix
% 
% %% \section{}
% %% \label{}
% 
% %% References
% %%
% %% Following citation commands can be used in the body text:
% %% Usage of \cite is as follows:
% %%   \cite{key}         ==>>  [#]
% %%   \cite[chap. 2]{key} ==>> [#, chap. 2]
%%

% \twocolumn

%% References with BibTeX database:

% \bibliographystyle{elsarticle-num}
% \bibliographystyle{elsarticle-harv}
% \bibliographystyle{elsarticle-num-names}
% \bibliography{<your-bib-database>}
% \addcontentsline{toc}{chapter}{Bibliography}
\bibliographystyle{elsarticle-num}
% ``Overfull \hbox in .bbl'' message fixed by commenting out ''write.url'' in /usr/share/texlive/texmf-dist/bibtex/bst/elsarticle/elsarticle-num.bst for affected entry types (likely, 'article' and 'book', all but 'phdthesis')
% \bibliographystyle{ieeetr}
\bibliography{../../library}
% \thispagestyle{fancyTOC}



% H. F. Aly et al., Microchim. Acta, vol. 59, no. 1, 1971.

% K. S. Bhatki et al., J. Radioanal. Chem., vol. 2, no. 1-2, 1969.

% T. H. Bokhari et al., J. Radioanal. Nucl. Chem., vol. 283, no. 2, 2010.

% J. F. Briesmeister et al., Los Alamos National Laboratory, 1986.


% A. J. Koning et al., AIP Conference Proceedings, 2005, vol. 769, no. 2.

% H. Liskien et al., Nucl. Data Tables, vol 11, 1973.

% M. R. Lewis et al., J. Nucl. Med., vol. 44, no. 8, Aug. 2003.

% C. Müller et al., J. Nucl. Med., vol. 55, no. 10, Oct. 2014.

% N. Otuka et al., Nucl. Data Sheets, vol. 120, 2014.   (EXFOR)

% L. Pietrelli et al., J. Radioanal. Nucl. Chem., vol. 157, no. 2, 1992.

% S. M. Qaim et al., IAEA Technical Reports Series No. 473, 2011.

% T. Shimizu, et al, Ann. Nucl. Energy, vol. 31, no. 9, pp. 975-990, 2004.

% T. Shimizu, et al, Nucl. Instruments Methods Phys. Res. Sect. A Accel. Spectrometers, Detect. Assoc. Equip., vol. 527, no. 3, pp. 543-553, 2004.

% D. Updegraff et al., ''Nuclear Medicine without Nuclear Reactors or Uranium Enrichment,'' 2013.




%% Authors are advised to use a BibTeX database file for their reference list.
%% The provided style file elsarticle-num.bst formats references in the required Procedia style

%% For references without a BibTeX database:

% \begin{thebibliography}{00}

%% \bibitem must have the following form:
%%   \bibitem{key}...
%% 

% \bibitem{}

% \end{thebibliography}

\end{document}

%%
%% End of file `ecrc-template.tex'. 